
\chapter{Distributed algorithms simulation}

Be fore developing actual programs for wireless sensor networks, it
is good to check if the cooperation of local programs will lead
to working and efficient results.

The distributed behaviour comes from:

\begin{itemize}
\item an architecture specification, such as {\tt mekong1.occ},
\item a behaviour executed by nodes, such as {\tt nodes-test-include.occ}.
\end{itemize}

The two descriptions are orthogonal, meaning that one can make
evolution on the architecture at fixed behaviour, or make evolutions on
the behaviour with fixed architecture. The situation is well known
in computer architecture. It is named an Y methodological approach,
and was popularized by Gajski.
The bottom branch carrying measures produced from tools (energy, 
response time, cost, etc\ldots).

Simulation is a key method to take measures on complex situations.
To use simultation, we need to reproduce real dispersion of behaviours,
and random number generators are useful for this.

\section{Random numbers in Occam}

The Occam {\tt course} library provides the random function that
has two parameters and provides a result as a couple of numbers
(integers).

The small program {randomSample.occ}  demonstrates the general use for random.

\begin{lstlisting}  
#USE "course.lib"
-- montre le fonctionnement du generateur aleatoire.
PROC Random(CHAN OF BYTE in,out,err)
  VAL INT N IS 5: 
  VAL INT K IS 10:  -- borne de tirage de random : [0,K[
  PROC Genere(VAL INT seedInitial)
    INT x,seed:
    SEQ
      x,seed:=random(K,seedInitial)
      SEQ i=0 FOR N
        SEQ
          x,seed:=random(K,seed)
          out.number(x,2,out)
          out.string("*n",0,out)
      out.string("___",0,out)
      out.string("*n",0,out)
  :
  
  SEQ
    Genere(12)
    Genere(2)
    Genere(12) -- la souche est la meme que le premier cas et donne le meme tirage.
:
\end{lstlisting}

Compiling ({\tt kroc -lcourse randomSample.occ}), and executing  ({\tt ./randomSample})produces the 
following trace:
\begin{lstlisting}./randomSample
 1
 3
 1
 8
 0
___
 8
 9
 6
 3
 4
___
 1
 3
 1
 8
 0
___
\end{lstlisting}

It is noticeable that the seed value allows to reproduce exactly the same random sequence of numbers.
Thus, if we are to use this mechanism inside network simulators, we must vary the seed at each node.
The unique  {\tt Identity} of nodes is a good way to do this.