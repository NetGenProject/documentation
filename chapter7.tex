
\chapter{Image analysis} 
 
Creation of sensing machines working in the environment necessitate 
a validation of in the context of physical scenarios. Some of these 
scenario are flooding, insect clouds, pollutions, fires.

The focus of this chapter are tools and methods allowing to produce inputs
for physical  simulations. These simulations are compute intensive. It is expected
that they could be achieved using discrete computation models such as
cellular automata. It is expected that they will cooperate with network simulations
in various ways: production of stimuli on sensors, or dangers for the sensor network
itself.

The interest of image anlysis is to produce regions that share similar characteristics.
Analysis follows a conventional flow, starting from a picture coming from sources such as
photographies, maps, radar images, to obtain regions of interests, on which physical simultaion will
take place.


\begin{figure}[hbtp]
\begin{center} 
\includegraphics[width=10cm]{AhmedFire.png}
\caption{Fire simulation (or locust expansion) using Cellular Automata implementing  states such as:
vegetation, burning, ashes.}
\label{fig:AhmedFire}
\end{center}
\end{figure}

An example of processing on such regions are cellular automata
reproducing fire expansions of phenomenons consuming characteristics such as the existence od enough vegetation;;
Figure \ref{fig:AhmedFire} is extracted from a
movie demonstration where "fires" are started randomly to "eat" such vegetation report \cite{AhmedFire}. The
work was achieved on an Nvidia GPU.


\section{General flow}

Thus,   modeling  physical regions automatically, or semi-automatically, following physical process 
characteristics, is certainly critical to lead both physical and control network  simulations jointly,
and synchronously. One can think of this as a sampling technique of the physical process sharing
a clock with the sensing network. Cellular automata are one way to implement the real world simulation,
starting from initial states and regions.


A general flow to obtain such regions, is :
\begin{enumerate}
\item  preprocessing  images 
\item segmenting images into blocks
\item recognition of similar cells and grouping into regions
\item processing regions an obtaining skeleton images
\end{enumerate}

Following this flow, higher level objects can be obtained, still with geographical definition in the case of
maps, or satellite imagery.

\section {Image preparation }

In the cas of maps, geographic information is yet presented in a comprehensive way. However analyzing maps is still useful
to obtain information without the direct contact of an initial information system.
More difficult is the case of satellite or plane images, because the contract between objects necessitate
pre processing. Figure \ref{fig:sideBySide} shows an example of procesing achieved using common tools
for management of photographies. The initial view is a satellite image (left), while the right part displays an improved
view, with better contrasts.

\begin{figure}[hbtp]
\begin{center} 
\includegraphics[width=10cm]{SenegalSideBySide.png}
\caption{Modifying original image for better contrasts or coulour selections. Image is a satellite view from Google Maps.}
\label{fig:sideBySide}
\end{center}
\end{figure}

These representations are coming from a very common image processing software 
allowing to change contrasts and colour mapping to fit tne necessity of the recognition.
Image processing paramers show the Red Green Blue statistic values in the original
and the modified image, with a better use of the value range in the second case
(Figure \ref{fig:colours1+2}).



\begin{figure}[hbtp]
\begin{center}
\leavevmode 
\begin{minipage}{6cm}
\begin{center} 
\includegraphics[width=6cm]{senegalColours.png} 
\end{center}
\end{minipage}~~~~~~~~~\begin{minipage}{6cm}
\begin{center} 
\includegraphics[width=6cm]{senegalColours2.png} 
\end{center}
\end{minipage}


\caption{Colour parameters for the left original picture, and its adaptation (right)}
\label{fig:colours1+2}
\end{center}
\end{figure}
  
